\section{MODPATH 7 Example 2: Unstructured Grid, Steady-State Flow}

Example 2 simulates the same groundwater flow system as example 1 using MODFLOW-USG and MODFLOW-6 with an unstructured grid to refine the area near the well. The unstructured grid is based on the structured grid used in example 1 (referred to as the base grid). The areal grid is refined around the base-grid location that contains the well in example 1 (row 11, column 10). The base-grid cell containing the well is refined three levels, which means that the unstructured grid contains an 8 x 8 array of 64 cells in place of the original base-grid cell. The grid is smoothed so that it will meet the requirements of MODPATH for unstructured grids. The same areal grid is used for all model layers. The three-layer unstructured grid contains a total of 1,953 cells. The well is located in cell 1,623 in layer 3.

todo plot grid and boundary conditions

In simulation 2A a ring of 16 particles is placed around faces 1 through 4 at the mid- depth of layer 3 and then tracked backward to their recharge locations at the water table. For unstructured grid simulations, particle starting locations are specified by cell number rather than by layer, row, column.

\begin{StandardFigure}
	{fig:ex-prt-p02-paths}
	{../figures/ex-prt-p02-paths.png}
\end{StandardFigure}

Simulation 2B uses backward particle tracking to generate a capture area for the well. A 10 x 10 array of 100 particles is placed on faces 1 through 4 of cell 1,623, which contains the well. An additional 4 x 4 array of 16 particles is placed on the top of the cell. Particles are trackied backward using an endpoint analysis.

\begin{StandardFigure}
	{fig:ex-prt-p02-endpts}
	{../figures/ex-prt-p02-endpts.png}
\end{StandardFigure}